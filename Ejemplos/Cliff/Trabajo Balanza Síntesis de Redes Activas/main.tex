\documentclass[12pt,A4]{article}
\usepackage[utf8]{inputenc}%codification of the document
\usepackage[spanish]{babel}
\usepackage{fancyhdr}%pie pagina
\usepackage{lastpage}
\usepackage{graphicx} %agregar imagenes
\usepackage[document]{ragged2e} %alinear texto
\usepackage{amsmath} %insentar formulas
\numberwithin{equation}{section} % para lista las ecuaciones por cada seccion Ej 1.1
\usepackage{caption}
\usepackage[hidelinks]{hyperref}
\usepackage{float}
\usepackage{multirow} %generacion de tablas https://www.tablesgenerator.com/
%forma para generar nuevas subsubsub secciones con titlesec, 
%cambiando el formato de \paragraph 
\usepackage{titlesec}
\setcounter{secnumdepth}{4}
\titleformat{\paragraph}
{\normalfont\normalsize\bfseries}{\theparagraph}{1em}{}
\titlespacing*{\paragraph}
{0pt}{3.25ex plus 1ex minus .2ex}{1.5ex plus .2ex}
%%%%%
\usepackage[export]{adjustbox} %%posicionamiento de imagenes
\usepackage[shortlabels]{enumitem} % para enumerar con letras
%\begin{enumerate}[a)] % a), b), c), ...
%\item
%\end{enumerate}
\usepackage[table,xcdraw]{xcolor} %color en las celdas
\usepackage{pdfpages} % para importar pdf
% \includepdf[pagecommand=\subsection{blub}]{path.pdf} ingresa un pdf en una seccion 
% \includepdf[pagecommand={}]{path.pdf} ingresa un pdf sin perder el numero de pagina
% \includepdf[]{path.pdf} se pierde el numero de pagina
%%%%%
%%Para enumerar con puntos 
%% Ej : 1
%%      2
%%          2.1
%%          2.2
\renewcommand{\labelenumii}{\arabic{enumi}.\arabic{enumii}}
\renewcommand{\labelenumiii}{\arabic{enumi}.\arabic{enumii}.\arabic{enumiii}}
\renewcommand{\labelenumiv}{\arabic{enumi}.\arabic{enumii}.\arabic{enumiii}.\arabic{enumiv}}


\usepackage{chngcntr}           %lista las figuras dentro de una seccion
\counterwithin{figure}{section} %necesario para llevar la cuenta de figuras en una seccion
\usepackage{subcaption}         %para tener subfiguras
\pagestyle{fancy}
\fancyhf{}
\chead{Cátedra de Síntesis de Redes Activas }

\rfoot{Pagina \thepage \hspace{1pt} de \pageref{LastPage}}


%Here begins the body of the document
\begin{document}
\begin{titlepage}
\begin{figure}
    \centering
    \includegraphics[width = 8cm]{Media/imagenes/logounc.jpg}
\end{figure}
\newcommand{\HRule}{\rule{\linewidth}{0.5mm}}
\begin{center}
\textsc{\Large Síntesis De Redes Activas} \\ [0.5cm] 
%\textsc{\large } \\ [0.5cm]
%\textsc{\large Group Design Project} \\ [0.5cm]

\HRule \\ 
[.2cm]
\Huge{\bf{\Large Trabajo Practico }} \\
\textsc{\bf{\Large Diseño de balanza de uso comercial}} \\
[0.2cm]
\HRule 

\end{center}

\centering{
\begin{tabular}{rl}
{\bf Profesor Titular:} & {Dr. Ing. Ferreyra Pablo}\\
{\bf Profesor Adjunto:} & {Ing. Reale Cesar}\\
{\bf Ayudante alumno:}& {Lucas Heraldo Duarte}\\
%{\bf Supervisor:} & {Insert your supervisor here}
\\
{\bf Alumnos:} & {Alaniz Franco}\\& {Ferraris Domingo}\\
{\bf    Año Académico:} & {2021}\\
{\bf    Repositorio de GitHub:} & {https://github.com/DarioAlaniz/repoSistesisDeRedes}\\
%{\bf    Date:} & {14/06/2021}
\end{tabular}}

\end{titlepage}
\newpage
\tableofcontents
\newpage
\justifying
\section{Objetivo}

Diseñar la etapa analógica de una balanza comercial verificando requerimientos técnicos y competitividad de la misma en el mercado. Para tal fin, se aplicarán los conceptos teóricos y prácticos adquiridos a lo largo del semestre a un diseño real y concreto. Se implementarán, así, criterios propios y adquiridos para compatibilizar el desarrollo con los requerimientos del mercado.
\\La idea central yace en elaborar una balanza de calidad, con los mínimos costos posibles que pase todas las pruebas pertinentes para alcanzar su homologación y a su vez que pueda instalarse y competir con las marcas de balanzas líderes en el país como Systel y Kretz entre otras.
\section{Especificaciones}
\begin{itemize}
    \item Resolución : 1 g
    \item Temperatura de funcionamiento: 0$^\circ$C a 40$^\circ$C
    \item Rango máximo: 2 Kg
\end{itemize}
\begin{figure}[H]
    \centering
    \includegraphics[width=0.8\textwidth]{Media/imagenes/diagBloque.png}
    \caption{Esquema general}
    \label{fig:esqGeneral}
\end{figure}
\section{Marco Teórico}
\subsection{Celda de carga}
La celda de carga es un dispositivo capaz de traducir la fuerza aplicada sobre ella  en una señal eléctrica medible. En su interior están compuestas por galgas extensiométricas, una matriz de resistencias conectadas en forma de puente de Wheatstone, se observa en la \autoref{fig:celda}.
\begin{figure}[H]
    \centering
    \includegraphics[width=0.7\textwidth]{Secciones/marcoTeorico/media/celda.png}
    \caption{Celda de carga}
    \label{fig:celda}
\end{figure}
Donde la tensión de salida es:
\begin{equation}\label{eq:VoCelda}
    V_{out} = V_{in} \left(\frac{R1\cdot R3 - R2\cdot R4 }{(R1+R2)(R3+R4)}\right)
\end{equation}
Que cada vez que se coloca un elemento sobre ella, el bloque de alumino se deforma y en consecuencia las galgas se deformaran, variando así su resistencia eléctrica en proporción a la deformación producida. Esta variación será lineal siempre y cuando se trabaje en la zona lineal del material que conforma la galga de manera que la deformación que se produzca sea perfectamente elástica, es decir que la galga tenga la capacidad de volver a su forma original luego de la perturbación (ley de Hook).
\begin{figure}[H]
    \centering
    \begin{subfigure}[b]{0.49\textwidth}
        \includegraphics[width=\textwidth]{Secciones/marcoTeorico/media/galga1.png}
        \caption{}
        \label{fig:f1}
    \end{subfigure}
        \begin{subfigure}[b]{0.49\textwidth}
        \includegraphics[width=\textwidth]{Secciones/marcoTeorico/media/galga2.png}
        \caption{}
        \label{fig:f2}
    \end{subfigure}
    \caption{}
    \label{fig:galga}
\end{figure}
\subsection{Amplificación y Acondicionamiento}
Como la celda de carga emitirá una señal eléctrica de tipo diferencial que es proporcional a la deformación producida por el peso. Debido a las limitaciones de los ADC, esta señal para poder ser digitalizada, posteriormente procesada y mostrada al operador de la balanza deberá ser amplificada en un rango que cumpla con los requerimientos del conversor.
\\Debido a las exigencias de medida que imponen los sensores, se ha decidido conformar un amplificador de instrumentación a través de amplificadores operacionales individuales \autoref{fig:aoInstrument}. Estos tienen la particularidad de tener alta impedancia de entrada, baja impedancia de salida, ganancia variable, estable lineal y alta CMRR (hecho que nos favorece especialmente ya que nuestra señal de entrada es diferencial pura y queremos eliminar lo máximo posible el modo común).

\begin{figure}[H]
    \centering
    \includegraphics[width=0.7\textwidth]{Secciones/marcoTeorico/media/aoInstrument.png}
    \caption{Configuración del amplificador de instrumentación}
    \label{fig:aoInstrument}
\end{figure}
Donde V1 que será la tensión máxima que tenemos en el puente $(Vcc+ \Delta v)$ y V2 la tensión mínima$(Vcc- \Delta v)$.

La etapa de pre-amplificación aumenta la impedancia de entrada del conjunto y hace que se cancelen entre sí los posibles errores, sólo tendremos errores por características constitutivas propias de los amplificadores. Una buena forma de disminuirlos es utilizar una misma pastilla para desarrollar el amplificador de instrumentación ya que si todos los operacionales pertenecen a la misma pastilla sus errores tienden a ser iguales.

Profundizando sobre esto, analizaremos la salida en función de las entradas V2 y V1:
\\Obteniendo la tension en los punto A y B:
\begin{equation*}
    \begin{split}
        \frac{VA-V1}{RA} &= \frac{V1-V2}{RG} \Rightarrow
        VA = V1 \left(\frac{RA}{RG} + 1\right) - \left(\frac{RA}{RG}\right)V2\\
        \frac{V1-V2}{RG} &= \frac{V2-VB}{RB} \Rightarrow
        VB = V2 \left(\frac{RB}{RG} + 1\right) - \left(\frac{RB}{RG}\right)V1\\
    \end{split}
\end{equation*}
Restando ambas expresiones:
\begin{equation*}
    VB-VA = V2 -V1 \left(\frac{RA+RB}{RG} + 1 \right)
\end{equation*}
Donde el termino entre paréntesis representa la ganancia de la etapa pre-amplificadora, variando RG podemos variar dicha ganancia.
\\Analizando la etapa diferencial:
\begin{equation*}
    V o=-\frac{R 2}{R 1} V A+\left(1+\frac{R 2}{R 1}\right)\left(\frac{R 4}{R 3+R 4}\right) V B
\end{equation*}
Reemplazando VA, VB y considerando que $Vd=VB-VA$ y $Vcm=\frac{VA+VB}{2}$
\begin{equation}\label{eq:Vo}
    V o=-V d\left(\frac{1+\frac{R 2}{R 1}}{1+\frac{R 3}{R 4}}\left(\frac{1}{2}+\frac{R B}{R G}\right)+\frac{R 2}{R 1}\left(\frac{1}{2}+\frac{R A}{R G}\right)\right)+V c m\left(\frac{1-\frac{R 2 R 3}{R 1 R 4}}{1+\frac{R 3}{R 4}}\right)
\end{equation}
De la ecuación \autoref{eq:Vo} podemos notar que:
\begin{itemize}
    \item La ganancia en modo comun sera: 
    \begin{equation*}
        Ac = \left(\frac{1-\frac{R 2 R 3}{R 1 R 4}}{1+\frac{R 3}{R 4}}\right)
    \end{equation*}
    y deseamos que sea nula para ello \begin{equation*}
        \left(\frac{1-\frac{R 2 R 3}{R 1 R 4}}{1+\frac{R 3}{R 4}}\right) = 0 \Rightarrow \frac{R2}{R1} = \frac{R4}{R5}
    \end{equation*}
    \item Tomando: \begin{equation*}
        RB = RA \Rightarrow Ad = \frac{R2}{R1}\left(1+\frac{2RA}{RB}\right)
    \end{equation*}
    \item Si tenemos en cuenta los dos puntos anteriores, podríamos considerar que a priori la $RRMC \Rightarrow \infty$ valor deseado.
\end{itemize}
\section{Desarrollo}
\subsection{Selección de componentes}
\subsubsection{Alimentación}
Como es un producto comercial lo más conveniente es utilizar una batería de larga duración y
que sea accesible por cualquier persona y en cualquier lugar, por lo tanto, se decide utilizar
una batería de 9V marca Eveready \autoref{fig:pila}.
\begin{figure}[H]
    \centering
    \includegraphics[width=0.2\textwidth]{Secciones/Desarrollo/pila.png}
    \caption{Batería de 9V}
    \label{fig:pila}
\end{figure}

Es necesario utilizar otro elemento para entregar a la celda de carga la tensión necesaria para su funcionamiento.
Se elige utilizar una fuente de alimentación ajustable LM2596 \autoref{fig:fuente} para disminuir la tensión de 9V a 5V y entregar una tensión estable.
\begin{figure}[H]
    \centering
    \includegraphics[width=0.5\textwidth]{Secciones/Desarrollo/fuente.png}
    \caption{LM2596}
    \label{fig:fuente}
\end{figure}

\subsubsection{Celda de carga}
La celda de carga elegida es la modelo FH02 \autoref{fig:celda} con la principales características \autoref{fig:celdaData}, es una celda económica y cumple con los
requerimientos de diseño de nuestra balanza comercial.
\begin{figure}[H]
    \centering
    \includegraphics[width=0.5\textwidth]{Secciones/Desarrollo/celda.png}
    \caption{celda de carga:  FH02}
    \label{fig:celda}
\end{figure}
\begin{figure}[H]
    \centering
    \includegraphics[width=0.8\textwidth]{Secciones/Desarrollo/celdaData.png}
    \caption{}
    \label{fig:celdaData}
\end{figure}
A partir del dato de la tensión de salida nominal se puede calcular la tensión máxima de salida de la celda de carga. Teniendo en cuenta como se dijo anteriormente que se va a alimentar a la celda con 5V se obtiene dado por \autoref{eq:VoCelda}:
\begin{equation*}
    V_{out} = 5~V \cdot 0.2~\frac{mV}{V} = 11~mV
\end{equation*}
Que serán amplificados a un valor de 5 V previo a la lectura del ADC.
\subsubsection{Amplificador de instrumentación}
En la \autoref{fig:amplificadorInstru} se tiene el circuito amplificador que cuenta con 3 amplificadores operacionales, además tiene características tales como una alta impedancia de entrada y una impedancia de fuente que no afecta el cálculo de la ganancia. Como amplificador operacional se elige el modelo AD 8538.
En la \autoref{fig:amplificadorData} se pueden ver el AO y las especificaciones eléctricas principales
del mismo:
\begin{figure}[H]
    \centering
    \includegraphics[width=0.9\textwidth]{Secciones/Desarrollo/amplificadorInstru.png}
    \caption{Circuito Amplificador}
    \label{fig:amplificadorInstru}
\end{figure}
\begin{figure}[H]
    \centering
    \includegraphics[width=0.9\textwidth]{Secciones/Desarrollo/amplificadorData.png}
    \caption{AD 8538}
    \label{fig:amplificadorData}
\end{figure}
Como se mencionó anteriormente se requiere una salida de 5V a partir de una salida de la
celda de carga de 11 mV, entonces, se calcula la ganancia necesaria:
\begin{equation*}
    k = \frac{5 ~V}{11 ~mV} = 454.5
\end{equation*}

Para la obtención de los demás valores del circuito se decide fijar:
\begin{equation*}
    R1 = R2 = R3 = R4 = 2.2~k\Omega \Rightarrow R_f = 220~k\Omega
\end{equation*}
Mientras que para la obtención del valor de $R_g$ se prueba con distintos valores de la misma hasta encontrar la ganancia más cercana a la que se requiere(\autoref{fig:varRG}), encontrando un valor apto de:
\begin{equation*}
    R_g = 1 ~k\Omega
\end{equation*}
\begin{figure}[H]
    \centering
    \includegraphics[width=0.7\textwidth]{Secciones/Desarrollo/varRG.png}
    \caption{Variación de Rg}
    \label{fig:varRG}
\end{figure}

\subsection{Errores DC}
Del fabricante obtenemos:
\begin{equation*}
    \begin{aligned}
        &V_{os} = 13~uV\\
        &I_{os} = 50~pA\\
        &I_{pol+} = 50~pV\\
        &I_{pol-} = 0~pV\\
        &RRMC = 31.26 M\\
        &A_d = 11.22M\\
        &R = 125~\Omega\\
        &F_s = 5~V\\
        &V_{cm} = 2.5~V\\
    \end{aligned}
\end{equation*}
\subsubsection{Errores para el amplificador N°1}
\begin{equation*}
    \begin{aligned}
        &T_1=\left(-A_d\right) *\left(\frac{R_f+R_g}{2 R_f+R_g}\right)=(-11.22 M) * \frac{(220 k \Omega+1000 \Omega)}{(2 * 220 k \Omega+1000 \Omega)}=-5.62 ~M \\
        &A_{O L 1}=A_d=11.22 M \\
        &A_{C L 1}=\frac{A_{o L 1}}{1-T_1}=\frac{11.22 ~M}{1+5.62~ M}=2 \\
        &A_{C L 3}=-\frac{R_4}{R_2}=-\frac{2.2 k \Omega}{2.2~ k \Omega}=-1
    \end{aligned}
\end{equation*}
\paragraph{Error por tensión de offset}
    \begin{equation*}
        \Delta V_{o s}=V_{o s} * A_{C L 1} * A_{C L 3}=13 u V * 2 *(-1)=-25.94 ~\mu V
    \end{equation*}
\paragraph{Error por corriente de polarización}
    \begin{equation*}
        \Delta V_{I_{p o l}}=I_{p o l+} * \frac{R}{2} * A_{C L 1} * A_{C L 3}=50~ p A * \frac{125 \Omega}{2} * 2 *(-1)=-6.24 ~n V
    \end{equation*}
\paragraph{Error por ganancia de lazo abierto finita}
    \begin{equation*}
        \Delta V_{A_d}=\frac{F_s}{1-T_1} * A_{C L 3}=\frac{5 \mathrm{~V}}{1+5.62 \mathrm{M}} *-1=889.23~ \mathrm{nV}
    \end{equation*}
\paragraph{Error por RRMC finita}
    \begin{equation*}
        \Delta V_{R R M C}=\frac{V_{c m} * A_{c L 1}}{R R M C}=\frac{2.5 ~V * 2}{31.62 M}=157.76 ~\mathrm{nV}
    \end{equation*}
\paragraph{Error total}
    \begin{equation*}
        \Delta V_1=\Delta V_{o s}+\Delta V_{I_{p o l}}+\Delta V_{A_d}+\Delta V_{R R M C}=26.99 ~\mu \mathrm{V}
    \end{equation*}
\subsubsection{Errores para el amplificador N°2}
\begin{equation*}
    \begin{aligned}
        &T_2=\left(-A_d\right) *\left(\frac{R_f+R_g}{2 R_f+R_g}\right)=(-11.22 M) * \frac{(220~ k \Omega+1000 \Omega)}{(2 * 220 k \Omega+1000 \Omega)}=-5.62 ~M \\
        &A_{O L 1}=11.22 ~M \\
        &A_{C L 1}=\frac{A_{o L 1}}{1-T_1}=\frac{11.22 M}{1+5.62 M}=2 \\
        &A_{C L 3}=\left(\frac{R_3}{R_1+R_3}\right) *\left(\frac{1+R_4}{R_2}\right)=\left(\frac{2.2 k \Omega}{2.2 k \Omega+2.2 k \Omega}\right) *\left(1+\frac{2.2 k \Omega}{2.2 k \Omega}\right)=1
    \end{aligned}
\end{equation*}
\paragraph{Error por tensión de offset}
    \begin{equation*}
        \Delta V_{o s}=V_{o s} * A_{C L 1} * A_{C L 3}=13 u V * 2 * 1=25.94 ~\mu V
    \end{equation*}
\paragraph{Error por corriente de polarización}
    \begin{equation*}
        \Delta V_{I_{p o l}}=I_{p o l+} * \frac{R}{2} * A_{C L 1} * A_{C L 3}=50 p A * \frac{125 \Omega}{2} * 2 *1=6.24 ~n V
    \end{equation*}
\paragraph{Error por ganancia de lazo abierto finita}
    \begin{equation*}
        \Delta V_{A_d}=\frac{F_s}{1-T_2} * A_{C L 3}=\frac{5 \mathrm{~V}}{1+5.62 \mathrm{M}} * 1=889.23~ \mathrm{nV}
    \end{equation*}
\paragraph{Error por RRMC finita}
    \begin{equation*}
        \Delta V_{R R M C}=\frac{V_{c m} * A_{c L 1}}{R R M C}=\frac{2.5~ V * 2}{31.62 M}=157.76 ~\mathrm{nV}
    \end{equation*}    
\paragraph{Error total}
    \begin{equation*}
            \Delta V_2=\Delta V_{o s}+\Delta V_{I_{\text {pol }}}+\Delta V_{A_d}+\Delta V_{R R M C} = 26.99 ~u V
    \end{equation*}
\subsubsection{Errores para el amplificador N°3}
\begin{equation*}
        T_3=\left(-A_d\right) *\left(\frac{R_2}{R_2+R_4}\right)=(-11.22 M) * \frac{2.2~ k \Omega}{(2.2~ k \Omega+2.2~ k \Omega)}=-5.61 ~M \\
\end{equation*}
\paragraph{Error por tensión de offset}
\begin{equation*}
    \Delta V_{o s}=V_{o s} *\left(1+\frac{R_4}{R_2}\right)=13 ~u V *\left(1+\frac{2.2 ~k \Omega}{2.2 ~k \Omega}\right)=26 ~u V
\end{equation*}
\paragraph{Error por corriente de polarización}
\begin{equation*}
    \Delta V_{I_{p o l}}=\left(\frac{R_1 * R_3}{R_1+R_3}\right) * A_d * \frac{I_{o s}}{1-T_3} \approx R_3 * I_{o s}=2.2 ~k \Omega * 50 ~p A=110 ~\mathrm{nV}
\end{equation*}
\paragraph{Error por ganancia de lazo abierto finita}
\begin{equation*}
    \Delta V_{A_d}=\frac{F_s}{1-T_3}=\frac{5 ~\mathrm{V}}{1+5.62 \mathrm{M}} =891.25~ \mathrm{nV}
\end{equation*}
\paragraph{Error por RRMC finita}
    \begin{equation*}
        \Delta V_{R R M C}=\frac{F_s}{RRMC}= \frac{5~V}{31.26M} = 157.11 ~\mathrm{nV}
    \end{equation*}
\paragraph{Error total}
\begin{equation*}
    \Delta V_3=\Delta V_{o s}+\Delta V_{I_{\text {pol }}}+\Delta V_{A_d}+\Delta V_{R R M C} = 27.16 ~u V
\end{equation*}  
\subsubsection{Tabla resumen de Errores}
\begin{table}[H]
\centering
\caption{Resumen de errores.}
\label{tab:my-table}
\begin{tabular}{|c|c|c|c|}
\hline
\multicolumn{1}{|l|}{} & \cellcolor[HTML]{FFCC67}Amplificadores & \multicolumn{1}{l|}{\cellcolor[HTML]{FFCC67}} & \multicolumn{1}{l|}{\cellcolor[HTML]{FFCC67}} \\ \hline
\rowcolor[HTML]{FFCC67} 
\cellcolor[HTML]{F8FF00}Errores & Nº1 & Nº2 & Nº3 \\ \hline
\cellcolor[HTML]{F8FF00}$\Delta V_{o s}$ & -25.94 uV & 25.94 uV & 26 uV \\ \hline
\cellcolor[HTML]{F8FF00}$\Delta V_{I_{p o l}}$ & -6.24 nV & 6.24 nV & 110 nV \\ \hline
\cellcolor[HTML]{F8FF00}$\Delta V_{A_d}$ & -889.23 nV & 889.25 nV & 891.25 nV \\ \hline
\cellcolor[HTML]{F8FF00}$\Delta V_{R R M C}$ & 151.76 nV & 157.76 nV & 157.11 nV \\ \hline
\rowcolor[HTML]{34FF34} 
\cellcolor[HTML]{F8FF00}$\Delta V_x$ & \textbf{26.99 uV} & \textbf{26.99 uV} & \textbf{27,16 uV} \\ \hline
\end{tabular}
\end{table}

Tenemos un valor de error total de:
\begin{equation*}
    \Delta V = \Delta V_1 + \Delta V_2 + \Delta V_3 = 81.15 ~uV
\end{equation*}
\subsection{Errores debido a la temperatura}
\subsubsection{Drift de Ganancia}
La Ganancia del amplificador de instrumentación es:
\begin{equation*}
    A_{AI} = \left(\frac{R2}{R4}\right) \left(\frac{2Rf}{Rg} + 1\right)
\end{equation*}
Para evitar que esta ganancia varíe con los cambios de la temperatura se deben usar
resistencias con el mismo coeficiente de temperatura. En este caso se escogieron resistencias
MFS1/4D con un coeficiente de temperatura de 100 ppm/ºC sí casi no se introducen errores extras, al menos en este aspecto.
\subsubsection{Drift de Vos}
El voltaje de offset varía con la temperatura, según la hoja de datos del amplificador este
tiene un valor de:
\begin{equation*}
    \frac{\Delta V_{os}}{\Delta T} = 0.1 \frac{uV}{ºC}
\end{equation*}

Multiplicando por $T_{max} = 40ºC$ para la aplicación resulta:
\begin{equation*}
\Delta V_{o s}=0.1 \frac{\mu V}{{ }^{\circ} \mathrm{C}} * 40^{\circ} \mathrm{C}=4 \mu V \rightarrow \frac{\Delta V_{o s}}{V_{o s}}=0.31
\end{equation*}
Entonces por proporción los errores a la salida en DC debido a Vos, que se calcularon antes,
se incrementan un 31\% debido a la temperatura. Dicha variación se calcula para cada AO:

\begin{equation*}
    \begin{aligned}
    &\Delta V_{o s 1}=0.31 \Delta V_{o s 1(D C)}=0.31 * 25.94 \mu V=7.98 \mu V \\
    &\Delta V_{o s 2}=7.98 \mu V \\
    &\Delta V_{o s 3}=8 \mu V \\
    &\Delta V_{t o t}=23.96 \mu V
    \end{aligned}
\end{equation*}
\subsection{Número de bits del conversor}
Con el valor del fondo de escala y el error total se puede calcular la resolución necesaria del
conversor y se expresa de la siguiente manera: 
\begin{equation*}
    \begin{aligned}
    &\Delta V_{T O T}=\Delta V_{D C}+\Delta V_{T e m p}=81.15 \mu V+23.96 \mu V=105.11 \mu V \\
    &n=\log _2\left(\frac{F_S}{\Delta V_{T O T}}\right)=15.53 \text { bits } \rightarrow 15 \text { bits }
    \end{aligned}
\end{equation*}
Como se tiene un conversor con un número de bits mayor a 11 bits (2048) entonces este es
compatible con la resolución de 1 gramo provista en las especificaciones técnicas de la
balanza a diseñar.
\subsubsection{Simulaciones}

\begin{figure}[H]
    \centering
    \includegraphics[width=0.9\textwidth]{Secciones/Desarrollo/circ1.png}
    \caption{Circuito de simulación}
    \label{fig:circ1}
\end{figure}
\subsubsection{Simulaciones punto de DC para un puente balanceado}
\begin{figure}[H]
    \centering
    \includegraphics[width=0.8\textwidth]{Secciones/Desarrollo/simuDC.png}
    \caption{DC point}
    \label{fig:simuDC}
\end{figure}
Al no tener entrada diferencial, la salida distinta de cero corresponde a los errores de DC,
recordando que los errores producidos por el AO1 y AO2, son en realidad de igual magnitud y
de signo opuesto, por lo que solo quedan los errores producidos por AO3. Donde lo calculado
es muy similar a los visto en los resultados de la simulación.
\subsubsection{Simulaciones punto de DC para un puente desbalanceado}
\begin{figure}[H]
    \centering
    \includegraphics[width=0.9\textwidth]{Secciones/Desarrollo/circ2.png}
    \caption{}
    \label{fig:circ2}
\end{figure}
En la imagen de la izquierda se ve el puente con un desbalance debido a una deformación que
produjo un cambio de $1~m\Omega$ en las resistencias del puente, se observa una tensión diferencial. A la derecha se muestra la salida del amplificador de instrumentación.

\begin{equation*}
    \begin{aligned}
    &V_{\text {dif }}=\operatorname{sig}_{+}-\operatorname{sig}{ }_{-}=7.033348 \mu \mathrm{V}\\
    &V_{\text {out }}=3.1773741 \mathrm{mV}\\
    &A_{A I}=\frac{V_{\text {out }}}{V_{d i f}}=451.7
    \end{aligned}
\end{equation*}
\subsection{Calculo económico}
Los valores obtenidos se tomaron en dolares.
\begin{figure}[H]
    \centering
    \includegraphics[width=0.9\textwidth]{Secciones/Desarrollo/costos.png}
    \caption{}
    \label{fig:costos}
\end{figure}
Como se puede observar el valor total del material utilizado ronda los 14.5 dólares, este valor
multiplicado al precio del dólar blue de hoy (290) se transforma en 4060 pesos.

\section{Conclusión}
Como se puede observar en el apartado anterior se tiene un costo de producción de un valor
que ronda los 4060 pesos.
Los productos ofrecidos en el mercado con similares características se muestran a
continuación:
\begin{figure}[H]
    \centering
    \includegraphics[width=0.9\textwidth]{Secciones/Desarrollo/balanzas.png}
    \caption{}
    \label{fig:balanzas}
\end{figure}
\newpage
Como se puede observar, nuestro valor no esta muy alejado de lo que se encuentra hoy en día. Los costos podrían reducirse si se realiza una producción en masa compitiendo a un mas en el mercado.

\section{Bibliografía}
\begin{itemize}
    \item Apuntes de la materia.
    \item \url{https://www.sciencedirect.com/topics/engineering/metal-strain-gauge}
     \item \url{https://tml.jp/e/knowledge/strain_gauge/about.html}
\end{itemize}

\end{document}
